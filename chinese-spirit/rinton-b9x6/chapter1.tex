
\chapter{Background and Some Concepts} %Chapter 1

\section{Introduction} %1.1
We assume knowledge of general solid state physics as in Rosenberg. 
However, we shall begin by briefly summaring a few concepts that are 
basic to an understanding of later Chapters below.

\subsection{Elastic and plastic regims} %1.1.1
Immediately it is helpful to classify the discussion of mechanical
properties by defining two regimes (i) elastic and (ii) plastic.

\subsubsection{Elastic deformation} %1.1.1.1
The mechanical properties of materials are of vital importance 
in determining their fabrication and practical applications. Initially 
as a load is applied on the material, the nominal stress is defined as 
the load divided by the original cross section area, and the nominal 
strain as the extension divided by the original length. As the stress 
is increased, the strain increases uniformly and the deformation 
produced is completely reversible. This is so-called elastic region. 
The stress and resulting strain are proportional to one another and 
obey Hook's law.

From atomistic points of view, if we pull two atoms apart or push 
them together by a force, the atoms can find a new equilibrium 
position in which the atomic and applied forces are balanced. The 
force in the bond is a function of the displacement. The deformation 
of the bond being reversable means that, when the displacement 
returns to the initial value, so also does the force return 
simultaneously to its corresponding value. The bulk elastic 
behaviour of large solid bodies is the aggregate effect of the 
individual deformations of the bonds in them.

\begin{theorem}  %Theorem 1.1
This is the text for the theorem. This is the text for the
theorem. This is the text for the theorem. This is the text for the
theorem. This is the text for the theorem. This is the text for the
theorem. This is the text for the theorem. This is the text for the
theorem. This is the text for the theorem. This is the text for the
theorem. This is the text for the theorem. This is the text for the
theorem.
\end{theorem}

When the applied forces are sufficiently small, the elastic 
displacement is always proportional to force. This is Hook's law. 
The elastic constant is a key parameter, to express the coefficient 
of proportionality between force and displacement.

\begin{lemma}  %Lemma 1.1
This is the text for the lemma.  This is the text for the lemma.  This
is the text for the lemma.  This is the text for the lemma. This is
the text for the lemma. This is the text for the lemma. This is the
text for the lemma. This is the text for the lemma. This is the text
for the lemma. This is the text for the lemma. This is the text for
the lemma. This is the text for the lemma. This is the text for the
lemma. This is the text for the lemma.
\end{lemma}

When the applied forces are sufficiently small, the elastic 
displacement is always proportional to force. This is Hook's law. 
The elastic constant is a key parameter, to express the coefficient 
of proportionality between force and displacement.

\begin{proposition}  %Proposition 1.1
This is the text for the proposition. This is the text for the
proposition. This is the text for the proposition.  When the applied
forces are sufficiently small, the elastic displacement is always
proportional to force. This is Hook's law.  The elastic constant is a
key parameter, to express the coefficient of proportionality between
force and displacement.
\end{proposition}

When the applied forces are sufficiently small, the elastic 
displacement is always proportional to force. This is Hook's law. 
The elastic constant is a key parameter, to express the coefficient 
of proportionality between force and displacement.

\begin{example}  %Example 1.1
This is the text for the example. This is the text for the
example. This is the text for the example. This is the text for the
example. This is the text for the example. This is the text for the
example. This is the text for the example. This is the text for the
example. This is the text for the example. This is the text for the
example. This is the text for the example. This is the text for the
example.
\end{example}

When the applied forces are sufficiently small, the elastic 
displacement is always proportional to force. This is Hook's law. 
The elastic constant is a key parameter, to express the coefficient 
of proportionality between force and displacement.

\begin{corollary}  %Corollary 1.1
This is the text for the corollary. This is the text for the
corollary. This is the text for the corollary. This is the text for
the corollary. This is the text for the corollary. This is the text
for the corollary. This is the text for the corollary. This is the
text for the corollary. This is the text for the corollary. This is
the text for the corollary.
\end{corollary}

When the applied forces are sufficiently small, the elastic 
displacement is always proportional to force. This is Hook's law. 
The elastic constant is a key parameter, to express the coefficient 
of proportionality between force and displacement.

\begin{definition}  %Definition 1.1
This is the text for the definition. This is the text for the
definition. This is the text for the definition. This is the text for
the definition. This is the text for the definition. This is the text
for the definition. This is the text for the definition. This is the
text for the definition.
\end{definition}

When the applied forces are sufficiently small, the elastic 
displacement is always proportional to force. This is Hook's law.  
The elastic constant is a key parameter, to express the coefficient 
of proportionality between force and displacement.

When the applied forces are sufficiently small, the elastic 
displacement is always proportional to force. This is Hook's law. 
The elastic constant is a key parameter, to express the coefficient 
of proportionality between force and displacement.

\begin{proof}  %Proof.
This is the text for the proof. This is the text for the proof. This
is the text for the proof. This is the text for the proof. This is the
text for the proof. This is the text for the proof. This is the text
for the proof. This is the text for the proof proof.
\end{proof}

\subsubsection{Atomic forces and elastic properties} %1.1.1.2
Taking NaCl type ionic crystals as an example, Cottrell (1964a) 
discussed the interaction energy of a pair of univalent ions at a 
distance $r$ as
\begin{equation} %1.1
U(r) = \pm {e^2 \over r} + {B \over r^s},
\end{equation}
where $s \approx 9$, and where $+$ and $-$ refer to like and 
unlike ions respectively. Having summed the repulsive and 
attractive interactions with nearest neighbours, the total 
interaction energy of an ion can be written as
\begin{equation}
U_z = - A {e^2 \over r} + 6 {B \over r^s}
\end{equation}
where $A$ is called the Madelung constant, equal to 1.7476 for the 
NaCl type crystals. At the equilibrium condition, ${{dU_z} \over 
{dr}} = 0 $, at $r = r_0$. 

\noindent 
Thus,
\begin{equation}
 B = {{{Ae^2} r_{0}^{s-1}} \over {6s}}
\end{equation}
and  
\begin{equation} 
U_z =-{Ae^2\over r}\left[{1-\left({1\over s}\right)
\left({r_0\over r}\right)^{s-1}}\right]
\end{equation}
this is the work required to dissociate the crystal into 2N separate 
ions (N positive and N negative).

The elastic constant $E$,
\begin{equation} 
E = {f\over u} = {\left({1\over6}\right) \left({{\partial^2 U_z} 
\over {\partial r^2}}\right)_{r=r_0}} = {{(s-1) Ae^2} \over {6r_0^3}},
\end{equation}
where $U_z/6$ is the energy per each nearest-neighbour bond, 
and $u = r - r_0$, is the elastic displacement.

The bulk modulus of elasticity of the material is defined by
\begin{equation}
p = - K {{\Delta V} \over V }
\end{equation}
where $p$ is a hydrostatic pressure, ${{\Delta V} \over V}$ is the 
volume change.
\begin{equation}
K = {{-p} \over {\left({{\Delta V} \over V}\right)}} = -{{pr_0^2} \over 
{{r_0^2} {\left({{\Delta V}\over V}\right)}}} \cong 
{ f \over {{r_0^2} {\left({{3u} 
\over {r_0}}\right)}}} = { 1 \over {3r_0}} {\left({f \over u}\right)} 
= {{(s-1) Ae^2} \over {18r_0^4}}
\end{equation}

In KCl, it gives $K^T = 1.88 \times 10^{11}{\rm dyn \, cm^2}$, 
whereas the observed value (extrapolated to OK) is $2 \times 
10^{11}{\rm dyn \, cm^2}$. The corresponding calculations of elastic 
constants of metallic crystals are much more difficult for the laws 
of force are much complicated. we shall discuss this in Chapter 5.

\begin{table} %1.1
\tbl{Caption for Table~1.}{%
\begin{tabular}{@{}l@{\qquad\quad}l@{\qquad\quad}l@{\qquad\quad}l@{}}\toprule
Title 		&	Year	&  Author\\\colrule
X-ray photons	&	1912	&  M. von Laue\\
Electrons	&	1927	&  C. Davisson and L. H. Germer\\
Neutrons	&	1936	&  D. P. Mitchell and P. N. Powers\\\botrule
\end{tabular}
}
\end{table}

\subsubsection{Plastic deformation} %1.1.1.3  
Plastic deformation is characterized by a permanent deformation of 
the material. Unlike elastic deformation, it does not reverse on 
unloading but leave the material with a permanent shape. This is 
called the plastic region. Between these two regions, there is a 
limiting stress, called the yield stress of the material, or the 
critical resolved shear stress of it.                               

\section{Friction Mechanisms} %1.2
The mechanisms of friction are discussed in early books on the 
subject (e.g. Bowden and Taber, 1950). here we refer to the subsequent 
account of Stoneham {\it et~al.}  (1993). These workers note the following 
mechanisms: (i) Adhesion: surfaces adhere and then work is done in 
separating them. (ii) Ploughing: one surface pulls away small 
amounts of the other and (iii) Anelasticity: here the assumption is 
that energy is dissipated by dislocation motiion and plastic 
deformation in the material.

\begin{alphlist}
\item %(a)
Entry one.
\item %(b)
Entry two.
\end{alphlist}

We shall, in later chapters, discuss friction on a mesoscopic scale 
as well as specific atomistic studies. As to the first of these, we 
shall see below that two main steps are involved. The first of these 
is the characterization of rough surfaces and their contact. The 
second step is to invoke some law of friction. In such a law, we 
want here to lay stress on the central importance of atomic force 
microscope (AFM) data (see Appendix 2.1) and its interpretation.

Tribology, the study of surfaces in moving contact, is an important 
area for technology. In spite of this, friction, at the time of 
writing, is not well understood at an atomistic levels. Persson 
(1994) has posed some fundamental questions as follows:
\begin{enumerate}
\item%  (1) 
What is structure of sliding interface: both geometric and 
electronic?

\item%  (2) 
Where does the sliding take place?

\item%  (3) 
What is the physical origin of the sliding force?
\end{enumerate}

Person follows these somewhat general points with some more specific 
questions:

\begin{romanlist}
\item%  (i) 
Why is the frictional force F usually proportional to the load N?

\item%  (ii) 
What is the microscopic origin of `stick-and-slip' motion?
\item%  (i) 
Why is the frictional force F usually proportional to the load N?

\item%  (ii) 
What is the microscopic origin of `stick-and-slip' motion?
\item%  (i) 
Why is the frictional force F usually proportional to the load N?

\item%  (ii) 
What is the microscopic origin of `stick-and-slip' motion?
\end{romanlist}

The crystallinity of the structure is the prime cause of this 
behaviour, for it enables whole slabs of crystal to glide past one 
another. Each slip is a displacement, in certain glide direction, 
generally the crystal direction of closest atomic packing on certain 
crystal planes which is called slip planes. In fcc and hcp metals, 
these are mainly close-packed planes, but in bcc metals, the 
situation is complicated. It will be discussed later. Slip begins on 
some small area of the surface. The slip-front line between the 
slipped and unslipped areas is by definition a {\em dislocation} line. 
The glide motion of a dislocation is a property of a periodic crystal. 
The transition from the slipped to the unslipped region is spread over 
several atomic distances which is the width of the dislocation. Every 
atom in this transition region is pushed only a little further out of 
its original equilibrium site when it moves forward. This is the 
reason why dislocation can move easily in the crystal. Thus, the yield 
stress is much lower than the theoretical strength of crystals. 
Dislocation theory plays important role in understanding the 
microscopic processes in plastic deformation. Even the elastic theory 
of dislocation may explain many phenomena, such as yielding, work 
hardening, and etc. It provides not only a deeper qualitative physical 
picture of plastic deformation but also a certain degree quantitative 
analysis of it approximately.

Plastic deformation can also occur by twinning. The atoms slide, 
layer by layer to bring each deformed slab into mirror-image lattice 
orientation relative to the undeformed material. The critical stress 
of twinning is usually higher. Twins form at low temperature and 
rapid deformation, e.g. bcc iron strained quickly at room temperature 
and slowly at 100k.

\section{Griffith Criterion: Role of Surfaces} %1.3  
Griffith (1924) derived an expression for the elastic crack 
propagation on the basis of thermodynamic considerations. He reasoned 
that a crack would advance when the incremental release of stored 
elastic strain energy $dW_E$ in a body became greater than the 
incremental increase of surface energy $dW_s$ as new crack surface 
was created. For the two-dimentional case in plane stress
\begin{equation}
{W_E} = {{\pi \sigma^2 c^2} \over E}
\end{equation}
$$ {W_s} = 4c \gamma_s $$
where, $\sigma$ is the nominal stress; E, the elastic modulus; 2c, the 
length of the crack, and $\gamma_s$ the specific surface energy.

The Griffith criterion can then be written as
\begin{equation}
{\sigma_F} = {\sqrt {{2E \gamma_s} \over {\pi c}}}
\end{equation}
by the condition that
$${\partial \over {\partial c}} {W_E} \geq {{\partial W_s} \over 
{\partial c}}. $$

The Irwin analysis in fracture mechanics defined a paremeter, crack 
extension force, $G = {K^2} / E$ (in plane stress) is equal to a 
critical value, $G_{1c}$. the crack-resistance force of the material. 
For the elastic crack in an infinitely wide plate
\begin{equation}
{G_c} = {{K^2} \over E} = {{\sigma_F^2 \pi c} \over E}
\end{equation}
where $K$ is the stress intensity factor, $K_c$ is called fracture 
toughness. 

Comparing to Eq.~(1.2.2), $G_c = 2 \gamma_s$. The two approaches 
lead to the same result although their methods are different. The 
specific surface energy plays very important role in brittle fracture.

Engineering materials do not fracture in a completely elastic 
manner. The localized plastic deformation near crack tip gives the 
materials some toughness, or resistance to crack propagation. Orowan 
(1948) proposed to add a term $\gamma_p$, the plastic work expended 
during crack propagation to the elastic work $\gamma_s$ as an 
effective specific surface energy in Eq.~(1.2.2). The Griffith 
equation is modified to read (in plane stress)
\begin{equation}
\gamma_F = \sqrt {{{2E} \over {\pi c}} (\gamma_s + \gamma_p)} \sim 
\sqrt {{{2E \gamma_p} \over {\pi c}}}
\end{equation}
(Some authors wrote $2(\gamma_s + \gamma_p)$ as $2 \gamma_s + 
\gamma_p'$; then, $\gamma_p' = 2 \gamma_p$). From Eq.~(1.2.4), it 
seems $\gamma_s$ is no longer an important factor in this process. 
However, Tetelman (1963) showed that for the case of $\rm Fe-3\% Si$, by 
Frank--Read source multiplication,
\begin{equation}
\gamma_m = {\rm const.} \gamma_s N_0^{3\over2} 
\left({{v_0} \over {v_c}}\right)^2 T^{5 \over 2}
\end{equation}
where $\gamma_m$ is defined as the product of the work done in a unit 
volume element of material when the crack advances and the distance 
perpendicular to the crack in which the deformation is extensive. 
$N_0$ is the density of mobile dislocation sources, $v_0$ and $v_c$ 
are velocities of sound and the crack respectively. $\gamma_m$, like 
$\gamma_p$ is a measure of the intrinsic toughness of a solid. In 
Eq.~(1.2.5), $\gamma_s$ is a multiplying facter not an addition term. 
The change of $\gamma_s$ directly influences the change of $\gamma_m$.

Moreover, Lung and Gao (1985) calculated the relative $K_{ic}$ value 
of metals with a simplified dislocation motion model and BCS 
dislocation distribution function at the crack tip
\begin{equation}
G_c^p \cong 2 \gamma_p \propto W_i = E_0 (K_{ic}^0)^2 F_i 
(\theta_0) r_i^* (\theta_0)^{1\over2}
\end{equation}
where $K_{ic}^0$ is the fracture toughness in linear elastic case
($= \sqrt {E G_{ic}^0}$ or $\sqrt {2 \gamma_s E}$); ${r_i^*}$, the 
plastic zone size; and $E_0 \propto E^{-1}$. The $E_0$ in Eq.~(1.2.6) 
is proportional to the inverse of the elastic modulus of materials, 
and $F_i(\theta)$, the angular dependent function repectively.

Comparing Eq.~(1.2.6) with (1.2.5), the two approaches lead to 
the same conclusion that $\gamma_s$ plays the role as a multiplying 
factor in the expression of critical crack extension forces. For a 
multiplying factor,
\begin{equation}
{{\Delta (\gamma_s f)} \over {(\gamma_s f)}} = {{\Delta \gamma_s} 
\over {\gamma_s}} + {{\Delta f} \over f}.
\end{equation}

The relative change of $\gamma_s$ is as important as that of $f$.
If we consider the hidden role of atomic forces in the structure of 
dislocation core and dynamics, the role of interatomic force is not 
only in surface enery term but in the dislocation core structure also.

\section{Peierls Stress and Potential}  %1.4
A dislocation  experiences an oscillating potential energy as it 
glides in a crystal. In the Peierls model (Peierls, 1940), the bonds 
across the glide plane were considered to interact via an interatomic 
potential, while the remainder of the lattice was linearly elastic. 
Nabarro (1957) gave an analytical expression for the dislocation core 
model. One can approximately estimate the ideal lattice resistance 
to dislocation motion by means of the Peierls model. The resolved 
applied stress necessary to move the dislocation over the Peierls 
barrier is called Peierls stress, $\sigma_p$. The Peierls stress comes 
from the expression for the Peierls energy which changes for a 
translation of the dislocation by a distance smaller than the Burgers 
vector.

Figure~1.1, reproduced from Nabarro (1967), shows the Peierls model of 
a dislocation. The material above A and below B is regarded as forming 
an elastic continuum. The force between the rows A and B is a periodic 
function of the displacement.

\begin{figure}%1.1
%\epsfxsize=0.8\hsize
\ArtWork{fig-1.eps}
\caption{Caption for Fig.~1.}
\end{figure}

As the dislocation moves through the lattice, it passes through 
unsymmetrical configuration to a different symmetrical configuration 
in which one half plane of atoms on the expanded side of the glide 
plane lies midway between two half planes on the conpressed side. 
Further motion passes through unsymmetrical configurations back to a 
state equivalent to the original. The dislocation moves if a finite 
force acts on it. The critical stress is called Peierls stress. After 
a lengthy calculations,the approximate energy of misfit is given by
\begin{equation}
E=\left[{{b^2 \mu} \over {4 \pi (1 - \nu)}}\right] 
\left\{ 1 + 2 \cos 4 \pi \alpha \; \exp 
\left({{-4 \pi \zeta} \over b}\right) \right\}.
\end{equation}

The force acting on unit length of the edge dislocation is,
\begin{equation}
F=-\left({1\over b}\right){dE\over {d\alpha}} = {{2b \mu} \over {(1- \nu)}} 
\sin 4 \pi \alpha \; \exp \left({{-4 \pi \zeta} \over b}\right),
\end{equation}
where $\zeta = {a \over {2 (1- \nu)}}$ is a parameter measuring the 
width of dislocation. $\alpha b$, the displacement of the centre of 
the dislocation from the original equilibrium position and $\mu$, the 
shear modulus.

The maximum value of Eq.~(1.3.2) is the critical shear strength; the 
Peierls stress is given by
\begin{equation}
\sigma_p = { {2 \mu} \over {4 \pi (1- \nu)}} \; \exp \left({{-4 \pi \zeta} 
\over b}\right).
\end{equation}

Considering the spirit of this model, and extending the 
displacement of the centre of the dislocation to include the thermal 
vibration amplitude, the temperature dependence of the crss can be 
obtained (see Lung {\it et~al.}, 1966; or later Sec.~10.3).

%%%%end of file %%%%%%%%%%%%%%%




