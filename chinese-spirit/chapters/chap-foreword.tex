
\section[出版說明]{出版說明}

%《中國人的精神》內容簡介:
晚清以來,中國形象被嚴重扭曲。學貫中西、特立獨行的「老怪物」辜鴻銘,於1915年出版用英文寫成的《中國人的精神》,用自己的筆維護了中國文化的尊嚴,改變了部分西方人對中國的偏見。此書一出,轟動西方,後被譯為多種文字。  

在《中國人的精神》這本書中,作者把中國人和美國人、英國人、德國人、法國人進行了對比,凸顯出中國人的特征之所在:美國人博大、純樸,但不深沉;英國人深沉、純樸,卻不博大;德國人博大、深沉,而不純樸;法國人沒有德國人天然的深沉,不如美國人心胸博大和英國人心地純樸,卻擁有這三個民族所缺乏的靈敏;只有中國人全面具備了這四種優秀的精神特質。

辜湯生(1857年7月18日-1928年4月30日),字鴻銘,號立誠。祖籍福建省同安縣,生於南洋英屬馬來西亞檳榔嶼。學博中西,號稱「清末怪傑」,是中國近現代為數稀少的一位博學漢學中國傳統的同時,又精通西方語言與文化的學者。
他翻譯了中國「四書」中的三部——《論語》、《中庸》和《大學》,創獲甚巨;並著有《中國的牛津運動》(原名《清流傳》)和《中國人的精神》(原名《春秋大義》)等英文書,熱衷向西方人宣傳東方的文化和精神,產生了重大的影響,在西方形成了「到中國可以不看紫禁城,不可不看辜鴻銘」的說法。

%我們現在將英語作為一種``世界英語'' (World English) 來看待;
%於是,英語不再只是單純的一門異族語言,它同時融合者不同民族的表達方式並折射其多姿的文化。
%一個世紀以來,有過這樣的一位位中國人,他們以各自令人驚歎的完美英語,對世界解說的中國,
%對祖國表達這赤誠。
%如今,我們相信,還有更多的中國人胸懷一樣的向往,因為,跨實際的開放中國需要引進,也需要輸出。

%我們出版中國人的英語著述,正是為有志於此的英語學習者樹一個榜樣,
%為下個世紀的中國再添一份自信,還為世界英語的推廣呐一聲喊。

%選擇辜鴻銘的作品重排出版,當然不是宣揚他那不免乖張偏頗的行為思想,
%而是感動於他對中國傳統文化的奮力捍衛;驚歎於他那登峰造極、令人仰止的英語造詣。
%辜鴻銘通英法德俄等多種外語,但他的著述多用英文,而其中尤以《中國人的精神》為著。
%《中國人的精神》原載1914年的《中國評論》,1915年更名《春秋大義》在京出版,並很快被譯成德文,一時轟動西方。
%本書力闡中國傳統文化對於西方文明的價值,在當時中國文化面臨歧視、中華民族遭受欺淩的情況下,其影響尤為特殊。
%當然,對於我們現在的讀者,這首先該是一本極為寶貴的英語讀物。


\section[回憶辜鴻銘先生]{回憶辜鴻銘先生}

羅家倫

在清末民初一位以外國文字名滿海內外,而又以怪誕見稱的,那便是辜鴻銘先生了。
辜先生號湯生,福建人,因為家屬僑居海外,所以他很小就到英國去讀書,在一個著名的中學畢業,
受過很嚴格的英國文學訓練。這種學校對於拉丁文、希臘文,以及英國古典文學,都很認真而徹底地教授。
這乃是英國當時的傳統。畢業以後,他考進伯明罕大學學工程(有人誤以為他在大學學的是文學,那是錯的)。

回國以後,他的工程知識竟然沒有發揮的餘地。當時張之洞做兩湖總督,請他做英文文案。
張之洞當年提倡工業建設,辦理漢冶萍煤鐵等項工程,以「中學為體,西學為用」相號召,為好談時務之人。
他幕府裏也有外國顧問,大概不是高明的外國人士,辜先生不曾把他們放在眼裏。
有一天,一個外國顧問為起草文件,來向辜先生請問一個英文字用法。
辜默然不語,走到書架上抱了一本又大又重的英文字典,
碰然一聲丟在那外國顧問的桌上說:「你自己去查去!」這件小故事是蔡孑民先生告訴我的,
這可以看出辜先生牢騷抑鬱和看不起庸俗外國顧問的情形。

民國四年,我在上海愚園遊玩,看見愚園走廊的壁上嵌了幾塊石頭,刻著拉丁文的詩,說是辜鴻銘先生做的。
我雖然看不懂,可是心裏有種佩服的情緒,認為中國人會做拉丁文的詩,大概是一件了不得的事。
後來我到北京大學讀書,蔡先生站在學術的立場上網羅了許多很奇怪的人物。
辜先生雖然是老複辟派的人物,因為他外國文學的特長,也被聘在北大講授英國文學。
因此我接連上了3年辜先生主講的英國詩這門課程。我記得第一天他老先生拖了一條大辮子,是用紅絲線夾在頭發裏辮起來的,戴了一頂紅帽結黑緞子平頂的瓜皮帽,大搖大擺地走上漢花園北大文學院的紅樓,頗是一景。
到了教室之後,他首先對學生宣告:「我有三章約法,你們受得了的就來上我的課,受不了的就早退出:
第一章,我進來的時候你們要站起來,上完課要我先出去你們才能出去;
第二章,我問你們話和你們問我話時都得站起來;
第三章,我指定你們要背的書,你們都要背,背不出不能坐下。」
我們全班的同學都認為第一第二都容易辦到,第三卻有點困難,可是大家都J限於辜先生的大名,也就不敢提出異議。

3年之間,我們課堂裏有趣的故事多極了。
我曾開玩笑地告訴同學們說:「有沒有人想要立刻出名,若要出名,只要在辜先生上樓梯時,
把他那條大辮子剪掉,那明天中外報紙一定都會競相刊載。」當然,這個名並沒有人敢出的。
辜先生對我們講英國詩的時候,有時候對我們說:「我今天教你們外國大雅。」
有時候說:「我今天教你們外國小雅。」有時候說:「我今天教你們洋離騷。」
這「洋離騷」是什麼呢?原來是密爾頓(John Milton)的一首長詩Licidas.
為什麼Licidas會變「洋離騷」呢?
這大概因為此詩是密爾頓吊他一位在愛爾蘭海附近淹死亡友而寫成的。

在辜先生的班上,我前後背熟過幾十首英文長短的詩篇。在那時候叫我背書倒不是難事,最難的是翻譯。
他要我們翻什麼呢?要我們翻千字文,把「天地玄黃,宇宙洪荒」翻成英文,這個真比孫悟空戴緊箍咒還要痛苦。
我們翻過之後,他自己再翻。他翻的文字我早已記不清了,我現在想來,那一定也是很牽強的。
還有一天把他自己一首英文詩要我們翻成中文,當然我們班上有幾種譯文,最後他把自已的譯文寫出來了,
這個譯文是:「上馬複上馬,同我夥伴兒,男兒重意氣,從此赴戎機,劍柄執在手,別淚不沾衣,寄語越溪女,喝囑複何為!」
英文可能是很好,但譯文並不很高明,因為辜先生的中國文學是他回國後再用功研究的,雖然也有相當的造詣,卻不自然。
這也同他在黑板上寫中國字一樣,他寫中國字常常會缺一筆多一筆,而他自己毫不覺得。

我們在教室裏對辜先生是很尊重的,可是有一次,我把他氣壞了。
這是正當「五四」運動的時候,辜先生在一個日本人辦的
《華北正報》(North China Standard)裏寫了一篇文章,大罵學生運動,說我們這班學生是暴徒,是野蠻。
我看報之後受不住了,把這張報紙帶進教室,
質問辜先生道:「辜先生,你從前著的《春秋大義》(The Spirit of the Chinese People)我們讀了都很佩服,
你既然講春秋大義,你就應該知道春秋的主張是』內中國而外夷狄』的,你現在在夷狄的報紙上發表文章罵我們中國學生,是何道理?」這一下把辜先生氣得臉色發青,他很大的眼睛突出來了,一兩分鍾說不出話,
最後站起來拿手敲著講台說道:「我當年連袁世凱都不怕,我還怕你?」這件故事,現在想起來還覺很有趣味。
辜先生有一次談到在袁世凱時代他不得已擔任了袁世凱為准備帝制而設立的參政院的議員
(辜先生雖是帝制派,但他主張的帝制是滿清的帝制,不是袁世凱的帝制)。
有一天他從會場上出來,收到300銀元的出席費,他立刻拿了這大包現款到八大胡同去逛窯子。
北平當時妓院的規矩,是唱名使妓女魚貫而過,任押妓者挑選其所看上的。
辜先生到每個妓院點一次名,每個妓女給一塊大洋,到300塊大洋花完了,乃哈哈大笑,揚長而去。

當時在他們舊式社會裏,逛妓院與娶姨太太並不認為是不正當的事,所以辜先生還有一個日本籍的姨太太。
他是公開主張多妻主義的,他一個最出名的笑話就是:「人家家裏只有一個茶壺配上幾個茶杯,哪有一個茶杯配上幾個茶壺的道理?」
這個譬喻早已傳誦一時,但其本質確是一種詭辯。不料以後還有因此而連帶發生的一個引伸的譬喻。
陸小曼同徐志摩結婚以後,她怕徐志摩再同別人談戀愛,
所以對志摩說:「志摩!你不能拿辜先生茶壺的譬喻來作借口,你要知道,你不是我的茶壺,乃是我的牙刷,茶壺可以公開用的,牙刷是不能公開用的!」
作文和說理用譬喻在邏輯上是犯大忌的,因為譬喻常常用性質不同的事物作比,並在這裏面隱藏著許多遁詞。

辜先生英文寫作的特長,就是作深刻的諷刺。我在國外時,看見一本英文雜志裏有他的一篇文章,所采的體裁是歐洲中世紀基督教常用的問答傳習體(Catechism)。
其中有幾條我至今還記得很清楚,如:「什麼是天堂?天堂是在上海靜安寺路最舒適的洋房裏!
誰是傻瓜?傻瓜是任何外國人在上海不能發財的!什麼是侮辱上帝?
侮辱上帝是說赫德(Sir Roben Hart)總稅務司為中國定下的海關制度並非至善至美。」
諸如此類的問題有二三十個,用字和造句的深刻和巧妙,真是可以令人拍案叫絕。
大約是在1920年美國《紐約時報》的星期雜志上有一篇辜先生的論文,占滿第一頁全面。
中間插人一個辜先生的漫畫像,穿著前清的頂戴朝服,後面拖了一根大辮子。
這篇文章的題目是「沒有文化的美國」(The Uncivilized United States)。
他批評美國文學的盯候說美國除了Edgar Allan Poe所著的Annabelle Lee之外,沒有一首好詩。諸如此類的議論很多,可是美國這個權威的大報,卻有這種幽默感把他全文登出。
美國人倒是有種雅量,歡喜人家罵他,愈罵得痛快,他愈覺得舒服,只要你罵的技術夠巧妙。
像英國的王爾德、蕭伯納都是用這一套方法得到美國人的祟拜。
在庚子八國聯軍的時候,辜先生曾用拉丁文在歐洲發表一篇替中國說話的文章,使歐洲人士大為驚奇。
善於運用中國的觀點來批評西洋的社會和文化,能夠搔著人家的癢處,這是辜先生能夠得到西洋文藝界贊美佩服的一個理由。

無疑義的,辜先生是一個有天才的文學家,常常自己覺得懷才不遇,所以搞到恃才做物。
他因為生長在華僑社會之中,而華僑常飽受著外國人的歧視,所以他對外國人自不免取嬉笑怒罵的態度以發泄此種不平之氣。
他又生在中國混亂的社會裏,更不免憤世嫉俗。他走到舊複辟派這條路上去,亦是不免故意好奇立異,表示與眾不同。
他曾經在教室裏對我們說過:「現在中國只有兩個好人,一個是蔡元培先生,一個是我。
因為蔡先生點了翰林之後不肯做官就去革命,到現在還是革命。
我呢?自從跟張文襄(之洞)做了前清的官以後,到現在還是保皇。」這可能亦是他自己的解嘲和答客難吧!

